\newpage
\begin{center}
  \textbf{\large 3. Разработка и реализация}
\end{center}
\refstepcounter{chapter}
\addcontentsline{toc}{chapter}{3. Разработка и реализация}

\section{Общая архитектура программного решения}
\section{Взаимодействие модулей}
\section{Реализация визуального модуля}
\section{Реализация лидарного модуля}
\section{Механизм интеграции}