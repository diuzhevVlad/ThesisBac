\newpage
\begin{center}
  \textbf{\large 1. Аналитический обзор }
\end{center}
\refstepcounter{chapter}
\addcontentsline{toc}{chapter}{1. Аналитический обзор }

% \section{Существующие подходы визуальной одометрии}
% \section{Технологии одометрии и SLAM на базе лидаров}
% \section{Мультимодальные системы: общие принципы совмещения визуальных и лидарных данных}
% \section{Анализ современных решений и библиотек}

\section{Существующие подходы монокулярной визуальной одометрии и SLAM}
Задача визуальной одометрии (ВО) состоит в оценке траектории движения камеры 
$\mathcal{T}_{VO}$ путём анализа последовательности изображений 
$\{I_k\}_{0:K}$. SLAM (Simultaneous navigation and mapping) предполагает 
построение глобальной карты, а также проведение дополнительных оптимизаций траектории.
Сама траектория представляет из себя набор положений камеры 
в обозначенной глобальной системе координат:
\begin{equation}
    \mathcal{T}_{VO} = \{C_k\}_{0:K},
\end{equation}
где $C_k$ --- положение камеры в момент $k$. Чаще всего глобальная 
система координат выбирается согласно первому положению камеры в 
последовательности.

В качестве подзадачи рассматривается оценка пространственной трансформации 
между последующими положениями камеры $T_k^{k-1}$ на основе кадров $I_{k-1}$ 
и $I_k$.
\begin{equation}
    T_k^{k-1} = \begin{bmatrix}R_k^{k-1} & t_k^{k-1} \\ 0 & 1\end{bmatrix},
\end{equation}
где $R_k^{k-1} \in \mathbb{R}^{3\times 3}$, $t_k^{k-1} \in \mathbb{R}^3$ --- 
матрица поворота и вектор смещения между кадрами соответственно. Можем выразить 
положения камеры следующим образом:
\begin{equation}
    C_k = \prod\limits_{i=1}^k T_i^{i-1}
\end{equation}

Положим, что соседние кадры $I_{0}$ и $I_1$ содержат наборы соответствующих друг 
другу точек в однородной системе координат $\{p^1_i\} \in \mathbb{R}^3$ и 
$\{p^{0}_i\} \in \mathbb{R}^3$. Тогда, выполнено эпиполярное ограничение 
с нормализацией:
\begin{equation}
    p^{1\top}_i K_c^{-\top} E K_c^{-1} p^{0}_i=0,
\end{equation}
где $K_c \in \mathbb{R}^{3\times 3}$ --- матрица калибровки камеры (camera matrix), 
$E\in \mathbb{R}^{3\times 3}$ --- эссенциальная матрица (essential matrix).

Существуют методы, позволяющие оценить значение эссенциальной матрицы, на 
основе наборов точек $\{p^1_i\}$ и $\{p^{0}_i\}$. Наиболее популярными являются 
восьмиточечный и пятиточечный алгоритмы. Часто для улучшения качества оценки применяется RANSAC.

На основе полученной оценки эссенциальной матрицы возможно восстановить взаимную
ориентацию камер, а также направление смещения. Важно заметить, что в постановке 
задачи монокулярной одометрии невозможно оценить метрическое значение вектора смещения
без дополнительной информации. Кроме того, ориентация и направление восстанавливаются
не единственным способом:
\begin{equation}
R^{0}_1 = UWC^\top, t^{0}_1 = \pm u_3 \vee R^{0}_1 = UW^\top C^\top, t^{0}_1 = \pm u_3,
\end{equation}
где $U$, $V$ --- матрицы правых и левых сингулярных векторов в разложении $E$, $W = \begin{bmatrix}
    0 & -1 & 0 \\ 1 & 0 & 0 \\ 0 & 0 & 1
\end{bmatrix}$, $u_3$ --- третий столбец матрицы $U$. Т.к. значение длины вектора $t^{0}_1$
невозможно восстановить на данном этапе, обычно оно принимается равным $1$.

Так как полученная оценка неоднозначна, требуется дополнительная валидация. Наиболее
распространенным подходом является триангуляция с целью получения оценок 
пространственных координат соответствующих точек. Выбирается конфигурация, обеспечивающая
валидное положение точек относительно камеры, а именно --- точки должны находиться перед 
камерами (не иметь отрицательную глубину). Полученная конфигурация положений камеры и
точек в пространстве позволят построить базовую 3D карту.

Для сохранения единого масштаба, начиная с обработки 3 кадра применяются методы 2D-3D 
сопоставления при оценке последующих трансформаций (после первой итерации и триангуляции получены
3D координаты сопоставляемых точек). Наиболее широко используемыми являются методы 
EPnP, UPnP и OPnP. 

Выше указанные методы склонны к быстрому накоплению ошибок, что приводит к искажению 
масштаба и структуры. Возможным решением является применение алгоритма групповой корректировки
(Bundle Adjustment), решающего оптимизационную задачу минимизации ошибки перепроецирования 
путем восстановления одновременно положений камер и точек. Ранее оцененная структура
используется в качестве начальных условий. Такой подход позволяет значительно повысить
устойчивость и качество оценок, однако требует значительных вычислительных ресурсов. 
В связи с этим, групповая корректировка обычно применяется с частотой меньшей, чем предыдущие
шаги.

Рассмотренный выше подход является общим для большинства методов визуальной одометрии. 
При дальнейшем описании методов
сделан акцент на ключевые отличия или особенности, заключающиеся в 
деталях реализации таких как способы детекции и сопоставления статичных точек, 
восстановления масштаба с использованием семантической информации, модификации 
оптимизационных задач.

\subsection{Методы на основе ключевых точек}
Большую часть методов визуальной одометрии на данный момент составляют методы на 
основе ключевых точек. Идея состоит в детекции устойчивых точек, обладающих важной 
семантической или геометрической информацией и дальнейшее сопоставление их 
между кадрами. В остальном зачастую процесс следует общей методологии. 

Семейство методов ORB-SLAM --- это один из наиболее известных и эффективных методов визуальной 
одометрии и построения карт. Их подход к монокулярной одометрии включает в себя три основных модуля: 
\begin{enumerate}
    \item трекинг;
    \item создание карты;
    \item замыкание цикла.
\end{enumerate}
Трекинг осуществляется путем выделения бинарных ORB-дескрипторов на изображении и применении 
процедуры ассоциации данных на основе расстояния Хэмминга. Для задачи замыкания
цикла производится сопоставление наблюдаемых ключевых точек с сохраненными в карте.
С помощью метода Bag of Words проверяются гипотезы о совпадении. При обнаружении совпадения
производится глобальная оптимизация на позиционном графе --- данная процедура схожа с 
групповой корректировкой, однако оптимизация производится только по отношению к
положениям камеры, а не камеры и точек.

PLP-SLAM использует помимо визуальных признаков также геометрические в виде линий,
которые извлекаются с помощью LSD и преобразуются в LBD-дескрипторы. Таким образом
в ассоциации данных участвуют также признаки, обладающие важной геометрической информацией.
Кроме того, система становится более устойчива и модули локальной групповой корректировки, а также
замыкания цикла работают более эффективно.


\subsection{Прямые методы}
Прямые методы не используют дискретный набор точек-признаков. Вместо этого, их главная
идея заключается в оптимизации значения фотометрической ошибки на основе интенсивности точек при их проецировании на изображения. 
Таким образом прямые методы сводят задачу 2D-3D сопоставления к чисто оптимизационной.


Широко известным прямым методом является DSO (Direct Sparse Odometry). Суть заключается
в выборе подмножества точек с сильным градиентом интенсивности. Каждой точке 
присваивается значение глубины $d$. Выбирается последовательность ключевых кадров.
Общая функция ошибки по всем кадрам в скользящем окне и точкам:
\begin{equation}
    \mathcal{E} = 
    \sum_{i \in \mathcal{F}} \sum_{j \in \mathcal{N}(i)} 
    \sum_{{p}_i \in P_i}
    w_{i,j}({p}_i) \rho\!\Bigl(\,
    I_j(\pi({T}_j^i {x}_i)) - I_i({p}_i)
    \Bigr),
\end{equation}
где  $\mathcal{F}$ --- множество ключевых кадров в окне оптимизации, $\mathcal{N}(i)$ ---
множество соседних ключевых кадров к $i$, $P_i$ --- набор разреженных 
точек в кадре $i$ (по принципу сильного градиента), $w_{i,j}({p}_i)$ --- 
весовой коэффициент, $\rho(\cdot)$ --- робастная функция потерь (например, 
Хубера или Коши), $\pi(\mathbf{x})$ --- оператор перспективной проекции 
3D-точки $x$ в 2D-координаты на изображении. Путем последовательного решения оптимизационной
задачи $\min\limits_{d, T}(\mathcal{E})$ восстанавливаются глубины точек и взаимные расположения 
камеры. Данный подход демонстрирует высокую точность, сопоставимую с
методами на основе ключевых точек, однако требует значительных вычислительных ресурсов.
Кроме того, в окружениях с отсутствием текстурированных поверхностей прямые являются
более устойчивыми. К их слабым сторонам можно отнести чувствительность к резким изменениям интенсивности.

\subsection{Нейросетевые методы}
В последнее время широкую популярность набирают нейросетевые методы монокулярной
визуальной одометрии. Существует множество их разновидностей, предлагающих различные
способы интеграции нейросетей в данную задачу. Такие методы как SuperPoint SLAM, Droid slam
используют нейросетевые дескрипторы для определения целевых точек. В методе SuperVO используется
графовая нейронная сеть для построения ассоциаций между детектированными признаками. 
Некоторые подходы,такие как Dyna-VO используют нейросети в качестве модуля сегментации
движущихся объектов, что позволяет избегать формирования признака на них и значительно 
повысить робастность.

Особый интерес представляют нейросетевые методы, решающие задачу целиком. MagicVO обладает
сверточно-рекуррентной архитектурой с механизмами внимания, позволяющей эффективно 
анализировать визуально-временную информацию. Обучение предполагается проводить напрямую
на наборах данных с сопоставлением реального перемещения камеры. Стоит отметить, что
итоговый выход модели соответствует реальному масштабу в отличие от большинства рассмотренных
выше методов. 

\section{Технологии одометрии и SLAM на базе лидаров}
% \section{Постановка задачи одометрии на основе лидар}

Задача лидарной одометрии состоит в оценке траектории движения лидара 
$\mathcal{T}_{LO}$ путём анализа последовательности облаков точек 
$\{{P}_k\}_{0:K}$, полученных последовательно в моменты времени $k$:

\begin{equation}
    \mathcal{T}_{LO} = \{ L_k \}_{0:K},
\end{equation}
где $L_k$ --- положение сенсора в глобальной системе координат в момент $k$. Чаще 
всего, аналогично визуальной одометрии, за глобальную систему координат выбирается положение сенсора в 
момент начала записи траектории.

В качестве основной подзадачи лидарной одометрии выступает оценка пространственной 
трансформации между соседними положениями сенсора $T_k^{k-1}$ на основе облаков 
точек ${P}_{k-1}$ и ${P}_k$ и итоговые положения лидара формируются
следующим образом:
\begin{equation}
    L_k = \prod_{i=1}^{k} T_i^{i-1}
\end{equation}

Наиболее распространённый подход к оценке трансформации между облаками точек 
${P}_{k-1}$ и ${P}_k$ --- это алгоритм Iterative Closest Point (ICP).
Предполагая известными соответствия между точками двух облаков, 
ICP итеративно минимизирует ошибку расстояния между ними:
\begin{equation}
    E(R^{k-1}_k ,t) = \sum_{i=1}^{N}\|p_i^k - (R^{k-1}_k p_i^{k-1}+t)\|^2,
\end{equation}
где $p_i^k \in {P}_k$, а $p_i^{k-1} \in {P}_{k-1}$ --- соответствующие точки в соседних облаках.

Путем решения следующий оптимизационной задачи вычисляются оценки $R^{k-1}_k ,t^{k-1}_k$:
\begin{equation}
    \min_{R^{k-1}_k ,t^{k-1}_k}\sum_{i=1}^{N}\|p_i^k - (R^{k-1}_k p_i^{k-1}+t)\|^2
\end{equation}

Поиск соответствий --- нерешенная задача, в рамках которой ведутся исследования. 
В связи с этим часто используют приближённые соответствия ближайших соседей (nearest neighbors):
\begin{equation}
    p_i^{k-1} = \arg\min_{p \in {P}_{k-1}} \|p_i^k - p\|
\end{equation}

Следующим этапом производится фильтрация выбросов, где используются пороговые 
значения расстояния для отсеивания неверных соответствий:
\begin{equation}
    \|p_i^k - p_i^{k-1}\| < \tau
\end{equation}

Для решения оптимизационной задачи часто применяется метод сингулярного разложения (SVD). 
Определив центроиды двух наборов соответствующих точек:
\begin{equation}
    \mu_k = \frac{1}{N}\sum_{i=1}^{N}p_i^k, \quad \mu_{k-1} = \frac{1}{N}\sum_{i=1}^{N}p_i^{k-1},
\end{equation}
можем записать ковариационную матрицу:
\begin{equation}
    W = \sum_{i=1}^{N}(p_i^{k-1}-\mu_{k-1})(p_i^k-\mu_k)^\top,
\end{equation}
далее выполнив её сингулярное разложение:
\begin{equation}
    W = U\Sigma V^\top,
\end{equation}
решение для трансформации находится как:
\begin{equation}
    R^{k-1}_k = VU^\top, \quad t^{k-1}_k = \mu_k - R^{k-1}_k\mu_{k-1}
\end{equation}

Однако точность ICP сильно зависит от качества начальной 
оценки. Кроме того, метод склонен к застреванию в локальных минимумах. Для преодоления 
этих проблем используются методы извлечения геометрических признаков.

Важно отметить, что, облака точек, полученные вращающимся лидаром, искажаются движением сенсора. 
Поэтому перед применением ICP необходимо провести компенсацию движения (deskewing), 
которая обычно основана на модели постоянной скорости:
\begin{equation}
    p_{deskewed}(t_i) = T(t_i)^{-1} p_{raw}(t_i),
\end{equation}
где $T(t_i)$ --- трансформация, определённая для момента $t_i$ внутри одного оборота 
лидара, вычисляемая из предыдущей оценки скорости.

Для уменьшения накопления ошибок, аналогично визуальной одометрии, применяются
глобальные методы оптимизации, чаще всего на основе графов, с построением 
локальной или глобальной карты и замыканием цикла.

\subsection{Прямые методы (direct)}
Прямые методы напрямую вычисляют трансформации между двумя последовательными 
облаками точек без явного выделения ключевых признаков. Часто применяются усовершенствованные версии 
ICP, такие как Point-to-Plane ICP и Generalized ICP (GICP), которые учитывают 
геометрические свойства облаков точек и обеспечивают лучшую устойчивость 
и сходимость.

Метод Normal Distribution Transform (NDT) решает проблему соответствия точек 
иначе, используя вероятностный подход для вычисления 
трансформации. 

CT-ICP (Continuous-Time ICP) интерполирует позиции точек внутри облака между 
начальным и конечным положениями, позволяя эффективно оценивать непрерывную 
траекторию сенсора, что улучшает точность и устойчивость одометрии в динамичных 
условиях.

Современный подход KISS-ICP использует стандартную версию ICP c точечным сопоставлением 
(point-to-point) и адаптивным порогом для валидации соответствия точек, обеспечивая 
возможность применения с различными типами лидаров и условиями эксплуатации.


\subsection{Методы на основе ключевых точек (feature-based)}
Методы одометрии на основе ключевых точек извлекают характерные точки или 
признаки из облака точек для последующей оценки 
перемещения. Использование только выделенных признаков позволяет ускорить 
вычисления и повысить общую точность, устраняя шумы и выбросы, однако может привести
и к потере части значимой информации.

Широко известным подходом является LOAM (LiDAR Odometry and \\Mapping), который 
извлекает точки, расположенные на острых гранях и плоских поверхностях, вычисляя 
локальную гладкость поверхности. Затем производится сопоставление этих точек для 
оценки движения с помощью ICP. Дальнейшие модификации LOAM, такие как LeGO-LOAM, вводят 
сегментацию облака точек для классификации точек на землю и объекты, что улучшает 
точность выделения признаков.

Другим значимым представителем является SuMa (Surfel-based Mapping), использующий нормали 
поверхностей для одометрии. Предлагается сравнивать текущие облака точек с картой, 
представленной набором локальных элементов плоскостей. SuMa++ расширяет этот подход, вводя 
семантические метки, полученные при помощи RangeNet++, добавляя 
ограничения в процедуру ICP, что увеличивает робастность.

Современные SLAM-фреймворки, такие как GLIM и MOLA используют схожие подходы в модулях
лидарной одометрии.

\subsection{Нейросетевые методы}
Нейросетевые методы приобретают популярность благодаря возможности автоматического 
извлечения признаков и адаптации к разнообразным условиям. Эти подходы обычно 
используют глубокое обучение для выделения и сопоставления признаков между облаками точек.

LO-Net представляет собой нейросетевую архитектуру, предсказывающую нормали и 
динамические области, а также использующую пространственно-временные ограничения 
для геометрической согласованности последовательных облаков точек. 

Другой представитель --- LodoNet, где предлагается отбирать пары соответствующих ключевых 
точек и затем использовать модуль, вдохновленный PointNet, для оценки относительного положения. 

Подход Чо и др. использует обучение без учителя для сетей VertexNet и PoseNet, оценивая 
неопределённость точек и относительное положение между измеренными облаками, обеспечивая 
робастность и точность в условиях неопределённости.


\section{Мультимодальные системы: совмещение визуальных и лидарных данных}
Задача одометрии на основе слияния данных лидара и монокулярной камеры состоит 
в оценке траектории движения устройства путём совместного анализа 
последовательностей изображений и облаков точек. Основная сложность задачи 
заключается в эффективном объединении двух существенно различных типов данных.

\subsection{Методы со слабой связью (Loosely-coupled)}
Методы со слабой связью, такие как V-LOAM, сначала независимо получают оценки 
перемещения на основе каждого сенсора, а затем объединяют их или используют результаты 
работы одного алгоритма в качестве источника начальных условий или надстройки для другого. 
В частности, изображения используются для первичной оценки перемещения, а облака точек лидара 
используются для уточнения масштаба и коррекции дрейфа:
\begin{equation}
T_k^{k-1} = f(T_{k,\text{image}}, T_{k,\text{лидар}}),
\end{equation}
где $f(\cdot)$ --- функция объединения оценок с использованием, например, расширенного фильтра Калмана (EKF).

В настоящее время публикуется довольно мало работ посвященных методам со слабой связью,
т.к. данные методы зачастую опускают значимую часть взаимосвязей между данными. 

\subsection{Методы с сильной связью (Tightly-coupled)}
Методы с сильной связью интегрируют информацию непосредственно на уровне признаков 
и формируют единую задачу оптимизации. Например, метод LIMO объединяет изображения 
и облака точек в едином bundle adjustment, минимизируя совместную ошибку 
перепроецирования.


DVLO (Deep Visual-LiDAR Odometry) использует глубокую нейронную сеть с 
иерархическим многоуровневым объединением признаков. Для локального совмещения 
признаков изображения и облаков точек используются псевдоточки, которые 
формируются путём проекции на изображение. Локальное объединение описывается 
выражением:
\begin{equation}
F_i^L = \frac{1}{X}\left(F_i^c + \sum_{j=1}^{k} \sigma(\alpha s_{ij} + \beta) \cdot F_j^{pp}\right),
\end{equation}
где $F_i^L$ --- локальный признак, полученный совмещением, где $F_i^C$ --- признак изображения, 
$F_j^{pp}$ --- признаки псевдоточек, полученных из изображений, $s_{ij}$ --- 
схожесть признаков, $\alpha, \beta$ --- обучаемые параметры, а $\sigma$ --- сигмоидная функция.
Для глобального объединения используется цилиндрическая проекция облаков точек 
в псевдоизображения, затем выполняется адаптивная агрегация признаков:
\begin{equation}
F^G = \frac{A_P \odot F_P + A_L \odot F_L}{A_P + A_L},
\end{equation}
где $\odot$ --- поэлементное произведение, $F^G$ --- глобальный признак, $F_P$ и $F_L$ --- признаки из облаков точек и изображений 
соответственно, а $A_P, A_L$ --- адаптивные веса, рассчитанные нейронной сетью.
На основе глобальных признаков последовательных кадров производится оценка смещения
с помощью многослойного перцептрона. 

4DRVO-Net (4D Radar-Visual Odometry Network) использует аналогичный DVLO подход 
иерархической адаптивной агрегации признаков, однако вместо лидарных применяются 
радарные данные. Дополнительно учитывается информация о скорости объектов, что 
позволяет фильтровать динамические помехи и повышать устойчивость оценок. 
Структура сети позволяет эффективно объединять радарные данные с изображениями 
и минимизировать ошибки оценки траектории за счет итеративного уточнения на разных 
масштабах и уровнях признаков.

Основное направление развития мультимодальных методов визуально-лидарной одометрии
тесно связано с использованием нейросетевых архитектур.

% \section{Анализ современных решений и библиотек}

